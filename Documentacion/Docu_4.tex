
\documentclass{article}

\title{Proyecto con la Oficina de Responsabilidad Social Universitaria - PUJ, Cali. \\ Documento No.4\\ Procesos de Ingenier\'ia de Software \\ Juan Carlos Mart\'inez \\ } % Title

\author{Alejandro Cardona, Diego Lozada, Luis Osorio} % Author name

\begin{document}

\maketitle % Insert the title, author and date

\begin{tabular}{lr}
Fecha de elaboraci\'on: 10/10/2012 
\end{tabular}

\setlength\parindent{0pt} % Removes all indentation from paragraphs

\section{Reunion con el cliente}

El d\'ia mi\'ercoles 10/10/2012 se hace una reuni\'on con la clienta \'Angela Apellido, y se le generan preguntas acerca del proyecto con la oficina de responsabilidad social, las preguntas se presentan a continuaci\'on, y se muestran las respuestas y la respectiva socializaci\'on respecto a cada pregunta.

\begin{itemize}

\item Se pregunta acerca del Formato adecuado para el ingreso de los proyectos, seg\'un su tipo. En esta pregunta, se toman varios factores en cuenta, seg\'un los proyectos de Facultad o internos a la Universidad, el formato deber\'a generarse seg\'un la investigaci\'on de los grupos encargados en el curso sobre ese tema, tambi\'en se dice que para los proyectos externos, se investigaran documentos que la clienta entregara, para hacer un formato adecuado para el ingreso de estos documentos.

\item Se cuestiona acerca de quienes podr\'an ingresar los proyectos a el sistema. En este punto, se llega a la conclusi\'on de que los proyectos de las facultades, ser\'an ingresados, por los usuarios indicados internamente en la universidad para el ingreso de dichos proyectos.

Para los proyectos externos a la universidad, los encargados del registro en el formato, ser\'a el administrador, es decir, los ingresaran en la oficina de responsabilidad social.
\end{itemize}

En la reuni\'on se concuerda que el d\'ia mi\'ercoles 17/10/2012, se har\'a una reuni\'on grupal, para juntar las informaciones recogidas, realizar un mejor an\'alisis de los datos que se procesaran en el sistema, y ese mismo d\'ia en la tarde, generar preguntas mas concretas con la cliente. 

\end{document}
